{\Large Звук}
{
\definecolor{HumanAudioColor}{rgb}{0.2,0.2,0.6}
%
\begin{itemize}

\item Хоча звуки, океанічні хвилі та серцебиття не є електромагнітними, вони наведені на цій графіці як посилання на частоту. Інші властивості електромагнітних хвиль відрізняються від властивостей звукових хвиль.

\item Звукові хвилі обумовлені коливальним стисненням молекул. Звук не може рухатися у такому вакуумі, як космічний простір.

\item Швидкість звуку в повітрі на рівні моря - 1240 км / год (770mph).

\item Люди можуть чути лише звук між $\approx$20Hz і $\approx$20kHz.

\item Інфразвук (нижче 20 Гц) можуть відчувати внутрішні органи та почуття дотику.
%
%from http://www.worldtravelers.org/motionsickness.htm
Частота в діапазоні 0,2 Гц часто є причиною порушення функціонування вистибюлярного абарату (motion sickness).

\item Кажани можуть чути звук до $\approx$50kHz.

\item 88 клавіш для фортепіано шкали Equal Temperament точно розташовані на діаграмі частот.

\item Протягом століть люди прагнули розділити безперервний спектр звукової частоти на окремі музичні нотатки, які мають гармонічні стосунки. Музиканти з мікротоналу вивчають різні масштаби. Один останній список містить 4700 різних музичних шкал.

\item Музична нота A зображена на графіку як \ pscirclebox [fillstyle = твердий, fillcolor = HumanAudioColor] (\ textcolor (білий)} {A}}

\textcolor{gray}{\hrule}
\vspace{.1in}

%\item This image depicts air being compressed as sound waves in a tube from a speaker and then travelling through the tube towards the ear.\\
%%picture of sound
%\vspace{1.75in}
%\rput[t]{0}(1,-.5){\input{pictures/soundwave.tex}}
\end{itemize}
	\vspace{0.1in}
	%
	\begin{tabular}{rl}
	%\begin{minipage}[b]{2.45in}\rput(.65,1){\input{pictures/soundwave.tex}}
	\begin{minipage}[b]{2.45in}\rput(1.3,.55){\includegraphics[width=2.2in]{pictures/soundbmp.eps}}
	\end{minipage}&
	\hspace{0.0in}
	\begin{minipage}[b]{1.2in}\textcolor{white}{%
	Це зображення зображує повітря, що стискається як звукові хвилі в трубці з динаміка, а потім проходить через трубку до вуха.\\	}
	\end{minipage}
	\end{tabular}
	\vspace{.2in}
}
