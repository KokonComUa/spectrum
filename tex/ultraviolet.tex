{\Large Ультрафіолетове світло}
\begin{itemize}
\item Ультрафіолетове світло виходить за рамки людського бачення.

\item Фізики розділили діапазони ультрафіолетового випромінювання у вакуумний ультрафіолет (VUV), екстремальний ультрафіолет (EUV), далекий ультрафіолет (FUV), середній ультрафіолет (MUV) та біля ультрафіолету (NUV).

\item UV-A, UV-B та UV-C були введені в 1930-х роках Комісією Internationale de l'\ '(E) clairage (CIE, Міжнародна комісія з освітлення) для фотобіологічних спектральних смуг.

\item Короткочасна експозиція УФ-А викликає сонячний засмага, що допомагає захистити від сонячних опіків. Вплив УФ-В на користь людини, допомагаючи шкірі виробляти вітамін D. Надмірне ультрафіолетове опромінення викликає пошкодження шкіри. УФ-С шкідливий для людини, але використовується як герміцид.

\item CIE спочатку розділив UVA і UVB на 315nm, пізніше деякі фото-дерматологи розділили його на 320nm.

\item UVA підрозділяється (UVA1 і UVA2) тут є перемінний ефекту ДНК на 340 нм.

\item Сонце виробляє широкий спектр частот, включаючи весь ультрафіолетове світло, проте UVB частково відфільтровується озоновим шаром, а UVC повністю фільтрується земною атмосферою.

\item Шмель може побачити світло в діапазоні UVA, який допомагає їм ідентифікувати певні квіти.
\end{itemize}




% From: http://www.ping.at/cie/publ/abst/134-99.html
% Press Release:
% CIE Collection in Photobiology and Photochemistry, 1999
% CIE 134-1999 ISBN 3 900 734 94 1
% This volume contains short Technical Reports prepared by various Technical Committees within CIE Division 6.

% 134/1: TC 6-26 report: Standardization of the Terms UV-A1, UV-A2 and UV-B
% The terms UV-A, UV-B and UV-C were introduced in the 1930's by CIE Committee 41 on Ultraviolet Radiation as a short-hand notation for photobiological spectral bands. It was never intended that the bands were exclusive for different effects. The bands have been in widespread use in different medical fields and scientific research. UV-A and UV-B were divided at 315 nm by the CIE. In recent decades, some photo-dermatologists and others have used different dividing lines such as 320 nm without recognizing the importance of maintaining an international standardized terminology. Because the terminology is used in many fields, this report recommends that the 315 nm division between UV-A and UV-B be maintained. However, recent research has clearly shown a difference in the photobiological interaction of long and short wavelength UV-A radiation with DNA. This led to a further division of UV-A into UV-A1 and UV-A2 with a dividing line at approximately 340 nm. While this division may be of value, the committee does not recommend officially to split UV-A into these two sub-bands at this time. Further research may justify a dividing line different from 340 nm in the future.
