{\psset{gradlines=100,
 fillstyle=gradient,
 gradangle=90,
 gradmidpoint=1.0,
 linewidth=0pt,
 linestyle=none,
 linearc=0pt}
{\Large Visible Spectrum \hspace{0.05in}
	\psframebox[fillstyle=none,linestyle=none]{\psset{xunit=2.8in}
		\multido{%
			\nStartColor=0.00+0.1,
			\nEndColor=0.1+0.1,
			\nLeftSide=0.00+0.10,
			\nRightSide=0.10+0.10}{8}{%
			\definecolor{StartColor}{hsb}{\nStartColor,1,1}
			\definecolor{EndColor}{hsb}{\nEndColor,1,1}
			\psframe[fillstyle=gradient,
				gradangle=90,
				gradbegin=StartColor,
				gradend=EndColor,
				gradmidpoint=1.0,
				linewidth=0pt,
				linestyle=none]
			(\nLeftSide,0)(\nRightSide,0.15)}
			}
}
\begin{itemize}

\item The range of EMR visible to humans is called ``Light". The visible spectrum also closely resembles the range of EMR that filters through our atmosphere from the sun.

\item Other creatures see different ranges of visible light; for example bumble-bees can see ultraviolet light and dogs have a different response to colours than do humans.

\item The sky is blue because our atmosphere scatters light and the shorter wavelength blue gets scattered the most. It appears that the entire sky is illuminated by a blue light but in fact that light is scattered from the sun. The longer wavelengths like red and orange move straight through the atmosphere which makes the sun look like a bright white ball containing all the colours of the visible spectrum.

\item Interestingly, the visible spectrum covers approximately one octave.

\item Astronomers use filters to capture specific wavelengths and reject unwanted wavelengths. The major astronomical (visual) filter bands are depicted as  \psframebox[fillstyle=none,linestyle=none]{\astrofilter{0.07,0.03}{X}}

\end{itemize}
}
