{\psset{gradlines=100,
 fillstyle=gradient,
 gradangle=90,
 gradmidpoint=1.0,
 linewidth=0pt,
 linestyle=none,
 linearc=0pt}
{\Large Visible Spectrum \hspace{0.05in}
	\psframebox[fillstyle=none,linestyle=none]{\psset{xunit=2.8in}
		\multido{%
			\nStartColor=0.00+0.1,
			\nEndColor=0.1+0.1,
			\nLeftSide=0.00+0.10,
			\nRightSide=0.10+0.10}{8}{%
			\definecolor{StartColor}{hsb}{\nStartColor,1,1}
			\definecolor{EndColor}{hsb}{\nEndColor,1,1}
			\psframe[fillstyle=gradient,
				gradangle=90,
				gradbegin=StartColor,
				gradend=EndColor,
				gradmidpoint=1.0,
				linewidth=0pt,
				linestyle=none]
			(\nLeftSide,0)(\nRightSide,0.15)}
			}
}
\begin{itemize}

\item Діапазон ЕМВ, видимий людям, називається "Світлом". Видимий спектр дуже нагадує весь діапазон ЕМВ, що фільтрується через нашу атмосферу від Сонця.

\item Інші істоти бачать різні діапазони видимого світла; наприклад, шмелі можуть бачити ультрафіолетове світло, і собаки мають іншу відповідь на кольори, ніж у людей.

\item Небо синє тому, що наша атмосфера розсіює світло, і коротша довжина хвилі синього розсіюється найбільше. Здається, що все небо освітлено блакитним світлом, але насправді світло розсіюється від сонця. Довші довжини хвиль, як червоний та оранжевий, рухаються прямо через атмосферу, що робить сонце схожим на яскраво-білий шар, що містить усі кольори видимого спектра.

\item Цікаво, що видимий спектр охоплює приблизно одну октаву.

\item Астрономи використовують фільтри для захоплення певних довжин хвиль і відхилення небажаних довжин хвиль. Основні астрономічні (візуальні) фільтрувальні смуги зображені як \psframebox[fillstyle=none,linestyle=none]{\astrofilter{0.07,0.03}{X}}

\end{itemize}
}
