{\Large Полярізація}
\begin{itemize}

\item Коли фотон (частинка світла) проходить через простір, його вісь електричних і магнітних коливань не обертається. Тому кожен фотон має фіксовану лінійну полярність десь між ними 0\degree\ до 360\degree. Світло може бути також циркулярно та еліптично поляризовано.

\item Some crystals can cause the photon to rotate its polarization.

\item Ресивер, що очікує поляризовані фотони не прийме фотонів іншої поляризації. (прикл.\ Приймачі супутникових антен мають положення горизонтальної та вертикальної полярності).

\item Поляризований фільтр (як окуляри Polaroid\texttrademark\) може бути використаний для демонстрації поляризованого світла. Фільтр пропускає лише фотони, що мають визначену полярність. Два фільтри що перекриваються  під прямим кутом майже повністю блокують світло; проте, третій фільтр поміщений між першими двома під кутом 45\degree\ поверне поляризацію світла і дозволить частині світла вийти через всі три фільтри.

\item Світло, що відбивається від електричного ізолятора, стає поляризованим. Рефлектори що проводять струм не поляризують світло.

% from Doug Welch
\item Мабуть, найбільш ''надійно" поляризоване світло - це веселка.

\item Місячне світло також слабо поляризоване. Ви можете протестувати це, переглянувши місячне світло через лінзу окулярів Polaroid\texttrademark, потім повертайте цю лінзу, місячне світло буде темніше та світліше.

\end{itemize}


