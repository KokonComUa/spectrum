%Emission and Absorption lines
{\Large Випромінення та Поглинання}
\begin{itemize}

%\item Люмінесцентна лампа використовує люмінесцентний матеріал (фосфор) для перетворення ультрафіолетового випромінювання з використанням фотолюмінесценції у видиме світло. Фосфор поглинає УФ-світло та виділяє видиме світло.

\item Так як ЕМ випромінення проходить через елементи, то хвилі певної довжини  поглинаються і деякі нові починають випромінюватись.Це поглинання та випромінювання створює характерні спектральні лінії для кожного елемента, які корисні для визначення складу віддалених зірок. Ці лінії використовуються для доведення величину червоного зсуву віддалених зірок.

\item Коли фотон вдаряє атом, він може поглинатися, якщо енергія еквівалентна. Енергетичний рівень електрона піднімається - по суті, утримуючи випромінювання. Коли виділяється енергія, створюється новий фотон певної довжини хвилі. Стрибок енергії - це дискретний крок, а в атомі існує багато можливих рівнів енергії.

\item Йохан Балмер (Johann Balmer) створив цю формулу, що визначає довжину хвилі фотонного випромінювання ($\lambda$); де $m$ це рівень початкової енергії електрону і $n$ це фінальний рівень енергії електрону:

\hspace{1in}$\lambda = 364.56nm \left( \frac{\D m^2}{\D m^2 - n^2} \right)$

\item Значна частина міжзоряної речовини зроблена з найпростішого атома водню. Нижче наведені лінії емісії та поглинання видимого спектра водню.

\end{itemize}

%what happens if the photons energy is slightly higher than required to elevate the electrons energy level? Where does the excess energy go after raising the electrons energy level?


%Start X, End X, Label, label offset
\newcommand{\absorbemit}[4]{
	\psframe(#1,.075)(#2,.165)
	\psline[linestyle=solid](#2,0)(#2,.083)
	\uput{2pt}[270](#2,#4){#3}
}
\vspace{0.2in}
\psframebox[fillstyle=none,linestyle=none]{
	\rput(0.1,0){
	%Start at 760nm (0in)
		\psset{xunit=.0120in,linestyle=none, linewidth=1pt, linecolor=Black, fillstyle=solid, fillcolor=Black,linearc=0}
{
%  Red 		760-620 nm
%  Orange 	620-570 nm
%  Yellow 	570-550 nm
%  Green 	550-470 nm
%  Blue 	470-440 nm
%  Violet 	440-380 nm
\psset{gradlines=100,fillstyle=gradient,gradangle=90,gradmidpoint=1.0,linewidth=0pt,linestyle=none}
\definecolor{StartColor}{hsb}{.0,1,1} \definecolor{EndColor}{hsb}{.1,1,1} \psframe[gradbegin=StartColor,gradend=EndColor](0,0)(165,.15) %Red to Orange
\definecolor{StartColor}{hsb}{.1,1,1} \definecolor{EndColor}{hsb}{.2,1,1} \psframe[gradbegin=StartColor,gradend=EndColor](165,0)(200,.15) %Orange to Yellow
\definecolor{StartColor}{hsb}{.2,1,1} \definecolor{EndColor}{hsb}{.38,1,1}\psframe[gradbegin=StartColor,gradend=EndColor](200,0)(250,.15) %Yellow to Green
\definecolor{StartColor}{hsb}{.4,1,1} \definecolor{EndColor}{hsb}{.5,1,1} \psframe[gradbegin=StartColor,gradend=EndColor](250,0)(305,.15) %Green to Blue
\definecolor{StartColor}{hsb}{.5,1,1} \definecolor{EndColor}{hsb}{.8,1,1} \psframe[gradbegin=StartColor,gradend=EndColor](305,0)(380.25,.15) %Blue to Violet
}
		\absorbemit{-2}{103.715}{$H_\alpha$}{0}
		\absorbemit{103.715}{273.867}{$H_\beta$}{0}
		\absorbemit{273.867}{325.953}{$H_\gamma$}{0}
		\absorbemit{325.953}{349.826}{$H_\delta$}{0}
		\absorbemit{349.826}{363.03}{$H_\epsilon$}{0}
		\absorbemit{363.03}{371.14}{$H_\zeta$}{-.13in}
		\absorbemit{371.14}{376.5}{$H_\eta$}{.32in}
		\absorbemit{376.5}{380.25}{$H_\theta$}{0}
		\absorbemit{380.25}{385}{}{0}
{
	\psset{linearc=2pt,linecolor=white,linestyle=solid,linewidth=1pt,fillstyle=none}
	\uput{2pt}[180](70,.17){Emission line}\psline{->}(70,.17)(86,.17)(86,.12)(102,.12)
	\uput{2pt}[180](70,-.10){Absorption line}\psline{->}(70,-.10)(86,-.10)(86,.04)(102,.04)
	\uput{2pt}[0](140,-.16){Balmer series name}\psline{->}(140,-.16)(127,-.16)(127,-.09)(113,-.09)
}
	}
}
\vspace{0.2in}

%longer at opposite end, must use negative co-ordinates (Ex. subtract from the beginning of red visible light 760nm )
%$H_\alpha$% 656.285nm  103.715
%$H_\beta$% 486.133nm 273.867
%$H_\gamma$% 434.047nm 325.953
%$H_\delta$% 410.174nm 349.826
%$H_\epsilon$% 396.97nm 363.03
%$H_\zeta$% 388.86nm 371.14
%$H_\eta$% 383.50nm 376.5
%$H_\theta$% 379.75nm 380.25

%from http://hyperphysics.phy-astr.gsu.edu/hbase/quantum/atspect.html#c1

%Emission
%Wavelength	Relative	Transition	Color
%(nm)		Intensity
%383.5384 	5 		9 -> 2 		Violet
%388.9049 	6 		8 -> 2 		Violet
%397.0072 	8 		7 -> 2 		Violet
%410.174 	15 		6 -> 2 		Violet
%434.047 	30 		5 -> 2 		Violet
%486.133 	80 		4 -> 2 		Bluegreen (cyan)
%656.272 	120 		3 -> 2 		Red
%656.2852 	180 		3 -> 2 		Red


\textcolor{gray}{\hrule}
\vspace{.4in}
\hspace{.3in}
\begin{tabular}{rl}
%blackbody radiation
\begin{minipage}[b]{1.1in}
	\psframebox[linestyle=none]{
		%
		%The following picture was created using the above Plank formula in a 
		% spreadsheet (balmer_series.gnumeric) then plotted.
		%The vertical scale is log(x), temperatures are 2000K, 1000K, 700K.
		\rput[bl](0,0){\includegraphics{pictures/blackbody.eps}}
		%
		\psline[fillstyle=none, linearc=0]{<->}(1.1,0)(0,0)(0,.9)
		\rput[b]{90}(-.05,.45){Power}
		\rput[t](.5,-.05){Wavelength}
		\psset{border=1pt, bordercolor=Black, fillstyle=none, linearc=0,linecolor=green}
		\psline{o-}(.08,.6)(.5,.6)\uput{2pt}[0](.5,.6){White Hot}
		\psline{o-}(.12,.43)(.7,.43)\uput{2pt}[0](.7,.43){Red Hot}
		\psline{o-}(.18,.25)(.9,.25)\uput{2pt}[0](.9,.25){Hot}
		\psline{o-}(.7,0)(.88,.1)(1.05,.1)\uput{2pt}[0](1.05,.1){CMB}
	}
\end{minipage}&
\hspace{0.0in}
\begin{minipage}[b]{3in}
\begin{itemize}
\item Макс Планк (Max Planck) визначив зв'язок між температурою об'єкта та його профілем випромінювання; Де $R_\lambda$ сила випромінювання, $\lambda$ довжина хвилі, $T$ температура:

\hspace{.7in} $R_\lambda = \frac{\D 37418}{\D \lambda^5 \epsilon^{\left( \frac{\D 14388}{\D \lambda T}\D - 1 \right) }}$

\end{itemize}
\end{minipage}
\end{tabular}
\vspace{.15in}




%Fluorescent lamp 12,000K
%Sun 6,000K
%Incandescent lamp 3000K
%Cosmic Microwave Background radiation 3K

