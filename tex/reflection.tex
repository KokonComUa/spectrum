%Віддзеркалення
{
\psset{linestyle=solid}
\textcolor{white}{\Large Reflection}

\begin{itemize}
\item Віддзеркалення ЕМВ залежить від довжини хвилі, як це продемонстровано, коли видиме світло і радіохвилі відбиваються від об'єктів, які проникні для Х-променів. Мікрохвильові печі, які мають велику довжину хвилі в порівнянні з видимим світлом, відскакують від металевої сітки в мікрохвильовій печі, тоді як проміння світла проходять наскрізь.

\end{itemize}

\begin{tabular}{cl}
%Дзеркало
\begin{minipage}[b]{1.8in}\rput(0.97,.3){%Віддзеркалення
{
\psset{linestyle=solid}
\textcolor{white}{\Large Reflection}

\begin{itemize}
\item Віддзеркалення ЕМВ залежить від довжини хвилі, як це продемонстровано, коли видиме світло і радіохвилі відбиваються від об'єктів, які проникні для Х-променів. Мікрохвильові печі, які мають велику довжину хвилі в порівнянні з видимим світлом, відскакують від металевої сітки в мікрохвильовій печі, тоді як проміння світла проходять наскрізь.

\end{itemize}

\begin{tabular}{cl}
%Дзеркало
\begin{minipage}[b]{1.8in}\rput(0.97,.3){%Віддзеркалення
{
\psset{linestyle=solid}
\textcolor{white}{\Large Reflection}

\begin{itemize}
\item Віддзеркалення ЕМВ залежить від довжини хвилі, як це продемонстровано, коли видиме світло і радіохвилі відбиваються від об'єктів, які проникні для Х-променів. Мікрохвильові печі, які мають велику довжину хвилі в порівнянні з видимим світлом, відскакують від металевої сітки в мікрохвильовій печі, тоді як проміння світла проходять наскрізь.

\end{itemize}

\begin{tabular}{cl}
%Дзеркало
\begin{minipage}[b]{1.8in}\rput(0.97,.3){%Віддзеркалення
{
\psset{linestyle=solid}
\textcolor{white}{\Large Reflection}

\begin{itemize}
\item Віддзеркалення ЕМВ залежить від довжини хвилі, як це продемонстровано, коли видиме світло і радіохвилі відбиваються від об'єктів, які проникні для Х-променів. Мікрохвильові печі, які мають велику довжину хвилі в порівнянні з видимим світлом, відскакують від металевої сітки в мікрохвильовій печі, тоді як проміння світла проходять наскрізь.

\end{itemize}

\begin{tabular}{cl}
%Дзеркало
\begin{minipage}[b]{1.8in}\rput(0.97,.3){\input{pictures/reflection.tex}}\end{minipage}&
\hspace{0.0in}\begin{minipage}[b]{1.8in}
	\textcolor{white}{ЕМВ будь-якої довжини хвилі може відображатися, однак
відбиваючий матеріал залежить від багатьох чинників, включаючи
довжину хвилі падаючого пучка.\vspace{3pt}\\
	Кут падіння ($\theta_i$) і кут відбиття ($\theta_r$) однакові.}
\end{minipage}\\
\end{tabular}
}



}\end{minipage}&
\hspace{0.0in}\begin{minipage}[b]{1.8in}
	\textcolor{white}{ЕМВ будь-якої довжини хвилі може відображатися, однак
відбиваючий матеріал залежить від багатьох чинників, включаючи
довжину хвилі падаючого пучка.\vspace{3pt}\\
	Кут падіння ($\theta_i$) і кут відбиття ($\theta_r$) однакові.}
\end{minipage}\\
\end{tabular}
}



}\end{minipage}&
\hspace{0.0in}\begin{minipage}[b]{1.8in}
	\textcolor{white}{ЕМВ будь-якої довжини хвилі може відображатися, однак
відбиваючий матеріал залежить від багатьох чинників, включаючи
довжину хвилі падаючого пучка.\vspace{3pt}\\
	Кут падіння ($\theta_i$) і кут відбиття ($\theta_r$) однакові.}
\end{minipage}\\
\end{tabular}
}



}\end{minipage}&
\hspace{0.0in}\begin{minipage}[b]{1.8in}
	\textcolor{white}{ЕМВ будь-якої довжини хвилі може відображатися, однак
відбиваючий матеріал залежить від багатьох чинників, включаючи
довжину хвилі падаючого пучка.\vspace{3pt}\\
	Кут падіння ($\theta_i$) і кут відбиття ($\theta_r$) однакові.}
\end{minipage}\\
\end{tabular}
}



