%Віддзеркалення
{
\psset{linestyle=solid}
\textcolor{white}{\Large Reflection}

\begin{itemize}
\item Віддзеркалення ЕМВ залежить від довжини хвилі, як це продемонстровано, коли видиме світло і радіохвилі відбиваються від об'єктів, які проникні для Х-променів. Мікрохвильові печі, які мають велику довжину хвилі в порівнянні з видимим світлом, відскакують від металевої сітки в мікрохвильовій печі, тоді як проміння світла проходять наскрізь.

\end{itemize}

\begin{tabular}{cl}
%Дзеркало
\begin{minipage}[b]{1.8in}\rput(0.97,.3){%Description: reflection schematic
{
\psset{linearc=0}
\psframe[fillstyle=solid,fillcolor=gray,linestyle=solid,linecolor=gray](-.8,.04)(.8,.1)	% Mirror body
\psline[linestyle=dashed,linecolor=gray](0,.1)(0,.7)
\psline[linecolor=gray](-.1,.1)(-.1,.2)(0,.2)
\psline[linecolor=white]{-}(-.8,0.1)(.8,0.1)	%Mirror surface
\rput(0,.1){
	\uput{4pt}[90](-.8;-30){\textcolor{white}{Source}}
	\psline[linecolor=white,fillstyle=none]{->}(-.8;-30)(0,0)(.8;30)
}
\rput(-.2,.5){\textcolor{white}{$\theta_i$}}
\rput(.2,.5){\textcolor{white}{$\theta_r$}}
\psarc[linecolor=white]{<->}(0,.1){.3}{30}{90}
\psarc[linecolor=white]{<->}(0,.1){.3}{90}{150}
\rput[t](0,0){\textcolor{white}{Reflector}}

}
}\end{minipage}&
\hspace{0.0in}\begin{minipage}[b]{1.8in}
	\textcolor{white}{ЕМВ будь-якої довжини хвилі може відображатися, однак
відбиваючий матеріал залежить від багатьох чинників, включаючи
довжину хвилі падаючого пучка.\vspace{3pt}\\
	Кут падіння ($\theta_i$) і кут відбиття ($\theta_r$) однакові.}
\end{minipage}\\
\end{tabular}
}



