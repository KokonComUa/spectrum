{\Large {\bfseries Е}лектро {\bfseries М}агнітне {\bfseries В}ипромінення (ЕМВ)}
\begin{itemize}
\item Електромагнітне випромінювання (ЕМВ) відбувається в дискретних одиницях, що називаються фотонами, але мають властивості хвиль, як це видно з наведених нижче зображень. ЕМВ може бути створено коливанням або прискоренням електричного заряду або магнітного поля. ЕМВ проходить через простір зі швидкістю світла (2.997 924 58 $\times 10^{8}$ \metrepersecond ). ЕМВ складається з коливального електричного та магнітного полів, які розташовані під прямим кутом один до одного і на певній довжині хвилі.
% From here: http://www.play-hookey.com/optics/basic_concepts/transverse_electromagnetic_wave.html
% There is some controversy about the phase relationship between the electrical and magnetic fields of EMR, one of the theoretical representations is shown here:
%Continuous electromagnetic radiation represented as a wave.

\rput[t]{0}(0.5,-.8){\input{pictures/emr.tex}}
\rput[t]{0}(2.8,-.8){\input{pictures/emr_particle.tex}}

\vspace{2.2in}

\item Природа частки ЕМВ можна побачити коли нонячний елемент перебуваючи під сильно затемненим світлом видає електрони поштучно.

\item Природа хвилі ЕМВ демонструє відомий експеримент з подвійною щілиною який показує гасіння та додавання хвиль.

\item Багато властивостей ЕМВ базований на теорії це по причині того що ми можемо лише ефекти ЕМВ а не сам фотом або саму хвилю.

\item Альберт Енштейн (Albert Einstein) висловив теорію, що швидкість світла швидша за все що може рухатись. До сих пір не змогли довести що це не так.

\item ЕМВ може змінити довжину хвилі, якщо джерело відступає або наближається, як у випадку червоного зсуву віддалених галактик і зірок, які віддаляються від нас на дуже високих швидкостях. Випущений спектральний світло з цих віддалених тіл виявляється більш червоним, ніж це було б, якби об'єкт не рухався від нас.

\item Ми маємо повний електронний контроль над частотами в діапазоні мікрохвиль та нижче. Вищі частоти повинні бути створені,  енергія має бути звільнена від елементів, таких як фотони. Ми можемо або накачувати енергію в елементи (наприклад, нагріта скеля з видимим полем ЕМВ, вивільняє інфрачервоне ЕМВ) або дозволити їй природно вивільнятись (наприклад, розпаду урану).

\item Ми можемо бачити тільки видимий спектр. Всі інші смуги спектру зображуються як штриховані кольори \psframebox[fillstyle=none,linestyle=none,framesep=0in]{\psframe[linearc=0,framearc=0,fillstyle=crosshatch,linewidth=0pt,linestyle=none, hatchwidth=2pt, hatchsep=1.5pt,hatchcolor=white](0,-.04)(.4,.1)}\hspace{0.4in}.


\end{itemize}
