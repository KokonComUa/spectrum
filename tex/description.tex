{
\psset{linearc=0}
{\Large How to read this chart}
\begin{itemize}

\item Ця діаграма організована в октавах (подвоєння / подвоєння частоти), починаючи з 1 Гц і вище (2,4,8 і т. Д.) І нижче (1/2, 1/4 і т. Д.). Октава - природний спосіб представлення частоти.

\item Частота збільшується по вертикальній шкалі в напрямку вгору.

\item Горизонтальні смуги обтікають з далекого права на крайній лівий, коли частота збільшується вгору.

%\item Ця діаграма також була спроектована таким чином, що секцію спектра можна було обрізати і обернути навколо стандартної 3-дюймової діафрагми довжиною 36 дюймів, щоб представляти безперервну частоту від низу до верху і зверху.

\item Однак немає обмежень на будь-який кінець цього графіка, однак через обмежений простір тут показані тільки «відомі» предмети. Частота 0 Гц є найнижчою можливою частотою, але метод відображення октав, що використовуються в цьому, не дозволяє назавжди досягаючи 0 Гц, тільки наближаючись до нього. Крім того, за визначенням частоти (Цикли в секунду) немає такої речі, як негативна частота.

\item Значення на діаграмі позначені наступними кольорами: \psframebox[framesep=1pt,fillstyle=solid,fillcolor=Black]{\textcolor{FColor}{Frequency}}  measured in Hertz, \psframebox[framesep=1pt,fillstyle=solid,fillcolor=Black]{\textcolor{WColor}{Wavelength}} вимірюється в метрах \psframebox[framesep=1pt,fillstyle=solid,fillcolor=Black]{\textcolor{EColor}{Energy}} вимірюється в електронВольтах.

\end{itemize}
}
