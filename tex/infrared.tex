{\Large Інфрачервоне випромінювання}
\begin{itemize}

\item Інфрачервоне випромінювання (ІЧ) відчувається людьми як тепло і знаходиться нижче діапазону людського зору. Люди (і що завгодно при кімнатній температурі) є випромінювачами ІЧ.

\item ІЧ сигнали дистанційного керування невидимі для людського ока, але можуть бути виявлені більшістю відеокамер.

\item Пристрої нічного бачення використовують спеціальну камеру, яка чутливо реагує на інфрачервоне зображення та перетворює зображення на видиме світло. Деякі ІЧ-камери використовують інфрачервону лампу, яка допомагає підсвітленню кадру.

\item ІЧ LASERи використовуються для пропалювання об'єктів.

\item Демонстрація IR полягає в тому, щоб тримати металеву чашу перед вашим обличчям. Інфрачервоне випромінювання, яке випромінює ваше тіло, буде відбито назад за допомогою параболічної форми чаші, і ви відчуєте тепло.

%Інформація надана Edward  Alan Dowdell:
%http://www.fiber-optics.info/fiber-history.htm
%http://www.thefoa.org/tech/wavelength.htm
%http://www.webopedia.com/TERM/E/EDFA.html
\item Інфрачервоні сигнали зв'язку на основі волоконно-оптичного зв'язку іноді посилюються ебрійо волоконно-оптичний підсилювачем  \scalebox{.8}{\psframebox[linestyle=none]{\fiberoptics{0.13,-.13}{EDFA}}}
\end{itemize}
