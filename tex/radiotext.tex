%Radio description
{\Large Радіо Діапазони}
%Draw a sinewave with
%	{Initial Amplitude x.xxx}{Initial Time-width x.xxx}{Initial Number of cycles x}
%	{Second Amplitude x.xxx}{Second Time-width x.xxx}{Second Number of cycles x}
%	{Total repeats}
\newlength{\nAmplitude}		\newlength{\nAmplitudeNeg}
\newlength{\nTime}		\newlength{\nTimeTwo}		\newlength{\nTimeThree}		\newlength{\nTimeFour}
\newlength{\nSAmplitude}	\newlength{\nSAmplitudeNeg}
\newlength{\nSTime}		\newlength{\nSTimeTwo}		\newlength{\nSTimeThree}	\newlength{\nSTimeFour}
\newlength{\nTimeEnd}		\newlength{\nSTimeEnd}		\newlength{\nSTimeStart}
\newcommand{\sinewave}[7]{
	\setlength{\nAmplitude}{#1}
	\setlength{\nAmplitudeNeg}{\nAmplitude*-1}
	\setlength{\nTime}{#2}
	\setlength{\nTimeTwo}{\nTime*2}
	\setlength{\nTimeThree}{\nTime*3}
	\setlength{\nTimeFour}{\nTime*4}
	\setlength{\nTimeEnd}{\nTime*4*#3}
	\setlength{\nSAmplitude}{#4}
	\setlength{\nSAmplitudeNeg}{\nSAmplitude*-1}
	\setlength{\nSTime}{#5}
	\setlength{\nSTimeTwo}{\nSTime*2}
	\setlength{\nSTimeThree}{\nSTime*3}
	\setlength{\nSTimeFour}{\nSTime*4}
	\setlength{\nSTimeEnd}{\nTime*4*#3+\nSTime*4*#6}
	\multido{\dXTimePosition=0.000in+\nSTimeEnd}{#7}{
		\multido{\dXPosition=\dXTimePosition+\nTimeFour}{#3}{
			\rput(\dXPosition,0){
			\parabola(0.00,0)(\nTime,\nAmplitude)\parabola(\nTimeTwo,0)(\nTimeThree,\nAmplitudeNeg)
			}
		}
		\setlength{\nSTimeStart}{\dXTimePosition+\nTimeEnd}
		\multido{\dXPosition=\nSTimeStart+\nSTimeFour}{#6}{
			\rput(\dXPosition,0){
			\parabola(0.00,0)(\nSTime,\nSAmplitude)\parabola(\nSTimeTwo,0)(\nSTimeThree,\nSAmplitudeNeg)
			}
		}
	}
}
%
\begin{itemize}

\item Радіочастотний спектр (ELF - EHF) заселеный набагато щільніше, ніж може бути показано на цій діаграмі. Було показано лише невелика вибірка частот, що використовуються у всьому світі.

\item Зв'язок з використанням ЕМВ виконується за допомогою:
\begin{itemize}
\item Амплітудної Модуляції (AM)
%Amplitude Modulation
\psframebox[linestyle=none,fillstyle=none]{\rput(.1,0.05){\sinewave{.1in}{0.015in}{5}{.05in}{0.015in}{5}{4}}}\vspace{0.1in}\\
\vspace{0.04in}OR
\item Частотної Модуляції (FM)
%Frequency Modulation
\psframebox[linestyle=none,fillstyle=none]{\rput(.1,0.05){\sinewave{.1in}{0.010in}{7}{.1in}{0.020in}{4}{4}}}
\vspace{0.1in}
\end{itemize}

\item Кожна країна має свої правила та розпорядження щодо виділення смуг у цьому регіоні. Зверніться до авторитету у вашому регіоні (наприклад, FCC в США, DOC в Канаді).

\item Не всі посилання погодяться на діапазон діапазонів ULF, тут використовується діапазон HAARP.

\item Радар використовує EMВ в діапазоні мікрохвиль, щоб визначити відстань та швидкість об'єктів.

\item В Канаді міське радіо містить 40 станцій між 26,965 - 27,405 МГц.

\item Шумановський резонанс утворюється в порожнині між Землею та іоносферою. Резонансні піки зображені як \psframebox[fillstyle=none,linestyle=none]{\blip{0,-.07}{S}}

\item Водневий газ випускає радіальний діапазон EMВ на 21 см \psframebox[fillstyle=none,linestyle=none]{\blip{0,-.07}{H}}

\item Деякі окремі частоти представлені у вигляді значків:\vspace{0.1in}\\
\begin{tabular}{cp{1.9in}cp{1.1in}}
\psframebox[linestyle=none]{\submarine{0.02,.05}{xxHz}}\hspace{0.2in}\vspace{0.05in} & Submarine communications&
\psframebox[linestyle=none]{\rput(0.15,.05)\pager}&Pager\\
\psframebox[linestyle=none]{\rput(0.15,.05){\timestandard}\hspace{.03in}}\vspace{0.05in} & Time / frequency standards&\psframebox[linestyle=none]{\rput(0.15,.05){\weatherstation}\hspace{.03in}\vspace{0.08in}} & Weather stations\\
\psframebox[linestyle=none]{\rput(0.14,0.01){\psframebox[fillstyle=solid,fillcolor=green,linecolor=green,linearc=0]{\textcolor{Black}{xxm}}}}\hspace{.1in}\vspace{0.05in} & Ham / international meter bands&
\psframebox[linestyle=none]{\rput(0.15,.04){
	\psframe[linestyle=solid,linecolor=gray,fillstyle=solid,fillcolor=gray,linearc=0](-.2,-.08)(.2,.08)
	\psframe[hatchwidth=2pt, hatchsep=1.5pt,linestyle=solid,linecolor=yellow,fillstyle=hlines,hatchangle=45,hatchcolor=yellow,fillcolor=gray,linearc=0](-.2,-.08)(.2,.08)
	}
	\hspace{.2in}}\vspace{0.05in}
 	& Short wave radio\\
\psframebox[linestyle=none]{\rput(0.05,.05){\psframebox[fillstyle=solid,fillcolor=Itinerant,linecolor=Itinerant,linearc=0,framesep=1pt]{\textcolor{Black}{GMRS}}}}&General Mobile Radio Service&
\psframebox[linestyle=none]{\rput(0.15,-0.05){\wirelessmic}}&Wireless Microphone\\
\psframebox[linestyle=none]{\rput(0.11,.05){\psframebox[fillstyle=solid,fillcolor=Itinerant,linecolor=Itinerant,linearc=0,framesep=1pt]{\textcolor{Black}{FRS}}}}&Family Radio Service&
\psframebox[linestyle=none]{\rput(0.15,.05){\psframebox[fillstyle=solid,linearc=0,linecolor=yellow,framesep=1pt,fillcolor=yellow,linewidth=1pt,linestyle=solid]{\textcolor{Black}{CP}}}\hspace{.03in}\vspace{0.08in}} & Cellular Phones\\
\psframebox[linestyle=none]{\rput(0.11,.05){\psframebox[linestyle=none,framesep=1pt,fillcolor=red,linearc=0]{\white SOS}}}&
	\psset{dotsize=1pt 1}Distress signal, in Morse code:\vspace{0.18in}
\rput(-1.4,-0.18){\rput(0,0.05){\psdots(0,0)(.1,0)(.2,0)\psline{cc-cc}(0.3,0)(0.42,0)\psline{cc-cc}(0.49,0)(0.61,0)\psline{cc-cc}(0.68,0)(0.8,0)\psdots(.9,0)(1,0)(1.1,0)}}&\hspace{0.1in}&\hspace{0.1in}\\
\end{tabular}

\end{itemize}
