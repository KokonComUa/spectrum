{\Large LASER (Лазер)}
\begin{itemize}

\item LASER це арконім з {\bfseries L}ight {\bfseries A}mplification by {\bfseries S}timulated {\bfseries E}mission of {\bfseries R}adiation.
\item Лазер це пристрій що продукує монохроматичне ЕМВ високої інтенсивності.
\item За наявності належного обладнання, будь-яке ЕМВ може працювати як лазерний пристрій. Наприклад, мікрохвилі використовуються для створення MASERа.

%\item Junk/reword: Фотони можуть бути об'єднані, і видно результуючу суму сил. Якщо поляризація та фаза однакові для фотонів, то отримане світло, імовірно, посилюється. Деякі ЛАЗЕРИ досягають цього завдяки постійному перекачуванню фотонів у фільтр, щоб отримати таку ж частоту, після чого відбивають ті відфільтровані фотони від протилежних дзеркал, щоб отримати правильну фазу та поляризацію. Як тільки світло буде "посилено", воно вийде з одного з дзеркал.

\end{itemize}
